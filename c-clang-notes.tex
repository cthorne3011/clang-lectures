\documentclass{article}

\title{Lecture notes on compiling C with Clang}

\begin{document}

\maketitle

[15/10/28]

Clang: compiler
LLVM: infrastructure

Read intro to LLVM

In x86 we say base pointer, but the general term is frame pointer

Automatic variables vs local variables?

static link works as an extra parameter for languages that have it

One frame per function call, not per function

Clang stack frame:

parameters slots are further down; we only use them if we need to
local vars allocated when running and placed at the end

[15/11/02]

Stack frame in x86:
- Why is return addr passed first? Because we use registers for parameters sometimes (only if necessary)
- Local variables furthest down/most recent because they are allocated by the function we call

Read LLVM calling conventions link

Clang function idiom:
L1. Push base pointer
L2. Adjust stack pointer (if used)
L3. Pop BP
L4. RET pops return addr

rax: accumulator

Stack frame are very cheap.
- just pushing, popping, moving to registers / manipulating pointers
- heap/garbage collection is much more costly

Computing the index in the frame
- How do we map (variable) names to indices in the stack frame?
- Symbol table

Keeping names for runtime slows things down, we prefer integer offsets

Compiled with clang -S
- Why does y have two values?
-- Parameters go into registers. If necessary we put them into the frame.
-- Parameters have a slot in the stack frame and are moved there if necessary.
-- Slots are always reserved for parameters. (ignoring some optimisation cases)

Leaf procedure: a procedure that doesn't call any other procedures
- they turn up as leaves if you draw the call tree
- leaf procedures don't need to use the stack pointer

[15/11/04]

Many arguments => spill into the stack
- We only have about 32 registers at most.
- Use them when we can but use stack if necessary.

printn/printf exercise

Last lecture we looked at leaf functions calls - but what about calling functions?

The order for computing argi is unspecified. Sometimes we swap the order for optimisation.
f(x++, x) - no guarantee that x++ will be done first

Sometimes we have to have parameters/values inside the frame. (i.e. when using ampersand/address operator)
- you can't point at a register using &
- ``escaping variables''

lea = load effective address (we want address, not value)

*p = x
=>
movq %rdi, (%rsi)
(``indirect addressing'')

callq can use *
callq *-8($rbp)

[15/11/11]

Constant propagation: e.g. if we have x=42, we can replace (some) occurrences of x with 42

Macros vs inlining:
- With macros, square(++y) would increment y twice

Careful identifying tail calls: return 5+f(E) is not a tail call
Tail calls can just use jmpq





\end{document}


